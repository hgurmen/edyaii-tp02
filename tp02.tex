\nonstopmode

\documentclass[a4paper,10pt]{article}

\usepackage[utf8]{inputenc}
\usepackage[english]{babel}
\usepackage{amsmath}

\author{
    Hernán Gurmendi \\
    \texttt{hgurmen@gmail.com}
    \and Nicolás Mendez Shurman \\
    \texttt{nicol12b@gmail.com}
}

\date{
    4 de junio de 2015
}

\title{
    \Huge Especificación de costos \\
    \Large Estructuras de Datos y Algoritmos II \\
    \large Trabajo práctico 2
}

\begin{document}

\maketitle

\begin{center}
\large \bf Docentes
\end{center}

\begin{center}
Mauro Jaskelioff

Cecila Manzino

Juan M. Rabasedas

Eugenia Simich
\end{center}

\newpage{}


\part*{Implementación con listas}


\section*{\texttt{filterS}}

Para la implementación de \texttt{filterS} con listas se usa la función
\texttt{filter} provista por el módulo \texttt{Prelude} ya que no se puede
paralelizar el cálculo. En este caso, lo único que se podría paralelizar es
el cálculo del predicado pasado como argumento.

% \; is a thick space
% \vert is |
% \left( is big left parenthesis
% \right) is big right parenthesis
% \sum is summatory.
% \max is max (math mode)
% \mathbf is bold formatting

\begin{equation*}
    W \left( filterS\; f \; s \right) \in
    O \left( \sum_{i=0}^{\vert s \vert -1} W(f \; s_i) \right)
\end{equation*}

\begin{equation*}
    S \left( filterS\; f \; s \right) \in
    O \left( \sum_{i=0}^{\vert s \vert -1} S(f \; s_i) \right)
\end{equation*}


\section*{\texttt{showtS}}

Hay que partir la lista dada en dos mitades, de modo que se necesita utilizar
las funciones \texttt{take} y \texttt{drop} definidas (que en el caso de la
implementación con listas son las mismas funciones que hay en \texttt{Prelude})
que tienen orden lineal. Esto resulta en costo lineal para trabajo y profundidad.

\begin{equation*}
    W \left( showtS \; s \right) \in
    O \left( \vert s \vert \right)
\end{equation*}

\begin{equation*}
    S \left( showtS \; s \right) \in
    O \left( \vert s \vert \right)
\end{equation*}


\section*{\texttt{reduceS}}

Para la implementación de \texttt{reduceS} con listas se usa la función
\texttt{contract} que tiene el siguiente trabajo y profundidad:


\begin{equation*}
    W(contract \; f \; s)
    \in
    O \left( \sum_{(x \oplus y) \in \mathcal{O}_r(\oplus,s)} W(x \oplus y) \right)
\end{equation*}

\begin{equation*}
    S(contract \; f \; s)
    \in
    O \left( \vert s \vert + \max_{(x \oplus y) \in \mathcal{O}_r(\oplus,s)} S(x \oplus y) \right)
\end{equation*}

\begin{equation*}
    W \left( reduceS \oplus b \; s \right) \in
    O \left( algo \right)
\end{equation*}

\begin{equation*}
    S \left( reduceS \oplus b \; s \right) \in
    O \left( algo \right)
\end{equation*}

\section*{\texttt{scanS}}

\begin{equation*}
    W \left( scanS \; s \right) \in O \left( algo \right)
\end{equation*}

\begin{equation*}
    S \left( scanS \; s \right) \in O \left( algo \right)
\end{equation*}

\newpage{}

\setcounter{section}{0}

\part*{Implementación con arreglos persistentes}

\section*{\texttt{filterS}}

Para la implementación con arreglos persistentes de \texttt{filterS} se aplica
la función \texttt{flatten} a una secuencia de tamaño $\vert s \vert$ donde cada
elemento de la misma es o un singleton o una secuencia vacía, de modo que el
trabajo de flatten resulta lineal y la profundidad logarítmica. Esta secuencia,
a su vez, se obtiene aplicando \texttt{tabulate} con una función (que tiene
trabajo y profundidad iguales a los del predicado f) y un tamaño $\vert s \vert$,
resultando entonces en los siguientes costos para \texttt{filterS}:

\begin{equation*}
    W \left( filterS \; f \; s \right) \in
    O \left( \sum_{i=0}^{\vert s \vert -1} W(f \; s_i) \right)
\end{equation*}

\begin{equation*}
    S \left( filterS \; f \; s \right) \in
    O \left( \text{lg} \; \vert s \vert + \max_{i=0}^{\vert s \vert -1} S(f \; s_i) \right)
\end{equation*}


\section*{\texttt{showtS}}

La implementación de \texttt{showtS} con arreglos persistentes usa las funciones
\texttt{takeS} y \texttt{dropS} (ambas con trabajo y profundidad O(1), ya que usan
\texttt{subArray}), de modo que resultan con los siguientes costos:

\begin{equation*}
    W \left( showtS \; s \right) \in
    O \left( 1 \right)
\end{equation*}

\begin{equation*}
    S \left( showtS \; s \right) \in
    O \left( 1 \right)
\end{equation*}


\section*{\texttt{reduceS}}

Para la implementación de \texttt{reduceS} con arreglos persistentes definimos
una función \texttt{contract} que contrae la secuencia dada con una operación
binaria utilizando la función \texttt{tabulate} provista, que a su vez utiliza
una función con el mismo costo que la operación binaria, de modo que el costo
de \texttt{contract} queda de la siguiente manera:

\begin{equation*}
    W \left( contract \oplus s \right) \in
    O \left( \sum_{i=0}^{\frac{\vert s \vert}{2} + 1} W \left( s_{2i} \oplus s_{2i+1} \right) \right)
\end{equation*}

\begin{equation*}
    S \left( contract \oplus s \right) \in
    O \left( \max_{i=0}^{\frac{\vert s \vert}{2} + 1} S \left( s_{2i} \oplus s_{2i+1} \right) \right)
\end{equation*}

Luego, \texttt{reduceS} utiliza \texttt{contract} en cada llamado recursivo,
dividiendo la entrada por 2 en cada paso. En el caso del trabajo, esto resulta
en recorrer la secuencia linealmente, además del trabajo de la operación binaria.
En el caso de la profundidad, \texttt{contract} paraleliza la aplicación de la
función binaria en cada nivel del árbol de reducción, luego la profundidad total
resulta en las profundidades más costosas de cada nivel (que es lg $\vert s \vert$).

\begin{equation*}
    W \left( reduceS \oplus b \; s \right) \in
    O \left( \vert s \vert + \sum_{(x \oplus y) \in \mathcal{O}_r(\oplus,b,s)} W \left( x \oplus y \right) \right)
\end{equation*}

\begin{equation*}
    S \left( reduceS \oplus b \; s \right) \in
    O \left( \text{lg} \; \vert s \vert \; \max_{(x \oplus y) \in \mathcal{O}_r(\oplus,b,s)} S \left( x \oplus y \right) \right)
\end{equation*}

\section*{\texttt{scanS}}

La implementación con arreglos persistentes de \texttt{scanS} también utiliza
la función \texttt{contract} definida por nosotros (cuyo costo está especificado
en el análisis de \texttt{reduceS}) y una nueva función \texttt{combine} que lo
que hace es combinar una secuencia ($s$) y el resultado de aplicar \texttt{scanS}
a la secuencia $s$ junto con la operación binaria y el elemento neutro ($s'$), de
manera de lograr la reducción buscada.

\texttt{combine} utiliza \texttt{tabulate} con una función que tiene costo $O \left( 1 \right)$
cuando el índice es par y $O \left( s'_{i} \oplus s_{2i-1} \right)$ cuando el índice 
es impar, de manera que el trabajo y profundidad dependen de éste último caso,
quedando:

\begin{equation*}
    W \left( combine \oplus s \; s' \right) \in
    O \left( \sum_{i=1}^{\frac{\vert s \vert}{2}} W \left( s'_{i} \oplus s_{2i-1} \right) \right)
\end{equation*}

\begin{equation*}
    S \left( combine \oplus s \; s' \right) \in
    O \left( \max_{i=1}^{\frac{\vert s \vert}{2}} S \left( s'_{i} \oplus s_{2i-1} \right) \right)
\end{equation*}

\begin{equation*}
    W \left( scanS \oplus b \; s \right) \in
    O \left( \vert s \vert + \sum_{(x \oplus y) \in \mathcal{O}_s(\oplus,b,s)} W \left( x \oplus y \right) \right)
\end{equation*}

\begin{equation*}
    S \left( scanS \oplus b \; s \right) \in
    O \left( \text{lg} \; \vert s \vert \; \max_{(x \oplus y) \in \mathcal{O}_s(\oplus,b,s)} S \left( x \oplus y \right) \right)
\end{equation*}

\end{document}
